\documentclass[11pt,a4paper]{article}
\usepackage{graphicx}
\usepackage{theorem}
\usepackage{color}
\usepackage{listings}
\copyright{GNU Licensed Document. Modifications and copies of this document
must follow the GNU License, refering to authors and to the original document.
All other rights are reserved by the authors.}

\title{MetaOS R integration. A how-to guide.}
\author{Sergio Alvarez (setelena@gmail.com),
    Luis F. Canals (luisf.canals@gmail.com)}

\begin{document}

\date{\today}
\maketitle


\section{Abstract}

\section{MetaOS viewed from Python scripts.}

The following example shows how Python scripts invokes a R code in 
the simplest way: load of R environment and R code, inizialization of R
variables and objects, example of loop to call several times to a function
R in the R-environment and new call at the end of the loop:

\lstset{language=Python,frame=single,tabsize=2,basicstyle=\tiny}
\lstinputlisting{../src/python/testR.py}

As it's seen from the code, a R source code named \emph{correlation.r}
where a class \emph{correlator} is defined. The class should have got the
methods \emph{memo} and \emph{show}, as in this example:

\lstset{language=R,frame=single,tabsize=2,basicstyle=\tiny}
\lstinputlisting{../src/R/correlation.r}



\section{Generalization}

The following code (\emph{rintegration.py}) uses the same described principle 
in the previous section but letting the name of R source containing the 
class as a runtime parameter:

\lstset{language=Python,frame=single,tabsize=2,basicstyle=\tiny}
\lstinputlisting{../src/python/testR.py}

In this case, interface for R class has been modified, to create a simple
predictor with two methods, \emph{learn} and \emph{predict}. An example of
predictor based on linear regression might look like this:

\end{document}
